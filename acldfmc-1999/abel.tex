\documentclass[twoside,twocolumn,a4paper]{article}
\usepackage{graphics}
\usepackage{hyperref}
\usepackage{amsmath}
\renewcommand{\columnsep}{10mm}
\textheight=24.3cm
\textwidth=16.0cm
\newcommand{\picwidth}{7.45cm}
\newcommand{\mpicwidth}{5.5875cm}
\newcommand{\spicwidth}{3.725cm}
\evensidemargin=-3mm %use as necessary for your printer
\oddsidemargin=0mm   %use as necessary for your printer
\topmargin=-17mm     %use as necessary for your printer
%\pagestyle{empty}
\pagestyle{plain}
\font\AFMCcl=cmssbx10 
\font\AFMCab=cmssbx10
\font\AFMCan=cmss10
\font\AFMChb=cmssbx10 scaled 900
\font\AFMCfg=cmss10 scaled 900
\font\AFMCnormal=cmr9
\font\AFMCfoot=cmr8
\setlength{\unitlength}{1in}

% find out if we're running PDFLaTeX or just plain old LaTeX so that
% we know whether to include EPS or PDF image files.
\newif\ifpdf
\ifx\pdfoutput\undefined
	\pdffalse	% we are not running PDFLaTeX
\else
	\pdfoutput=1	% we are running PDFLaTeX
	\pdftrue
\fi

% if we are using PDFLaTeX then include some PDF document information
\ifpdf
\pdfinfo {
	/Title		(INTERFEROMETRIC MEASUREMENT OF AN AXI-SYMMETRIC DENSITY FIELD)
	/Author		(A. F. P. Houwing, K. Takayama, K. Koremoto, T. Hashimoto, R. Faletic, M. Gaston)
	/Subject	(paper for the 2nd Australian Conference on Laser Diagnostics in Fluid Mechanics and Combustion, Monash University, Melbourne, Australia, 9-10 December 1999)
	/Keywords	(density, axi-symmetric, interferometer, phase, Fourier, Abel, deconvolutation, tomography)
}
\fi

\begin{document}

\setcounter{page}{111}

\title{
\sffamily \bfseries \large{INTERFEROMETRIC MEASUREMENT OF \\
AN AXI-SYMMETRIC DENSITY FIELD}}
\vspace{10mm}
\author{
\AFMCab \href{mailto:Frank.Houwing@anu.edu.au}{A.~F.~P. Houwing}, K.~Takayama\thanks{Shock Wave Research Center, Institute of Fluid Science, Tohoku University, Sendai, JAPAN}\ , K.~Koremoto$^*$, T.~Hashimoto$^*$, \href{mailto:Rado.Faletic@anu.edu.au}{R.~Faleti\v{c}}, \href{mailto:Matt.Gaston@eng.uts.edu.au}{M.~Gaston}\thanks{School of Aerospace and Mechanical Engineering, University College, University of New South Wales, Australian Defence Force Academy, Canberra, AUSTRALIA} \\
\AFMCan \\
\AFMCan \href{http://www.anu.edu.au/Physics/aldir/}{Aerophysics and Laser-based Diagnostics Research Laboratories} \\
\AFMCan \href{http://www.anu.edu.au/Physics/}{Department of Physics}, \href{http://www.anu.edu.au/science/}{Faculty of Science} \\
\AFMCan \href{http://www.anu.edu.au/}{Australian National University}, \href{http://www.canberratourism.com.au/}{Canberra}, \href{http://www.fed.gov.au/}{AUSTRALIA}
\vspace *{10mm}
}
\date{}
\maketitle
%\thispagestyle{empty}
\noindent
{\AFMChb ABSTRACT}
\begin{picture}(0,0)(0,0)
\put(-0.75,3.1){\small \it \href{http://www-mec.eng.monash.edu.au/ACLDFMC/}{2nd Australian Conference on Laser Diagnostics in Fluid Mechanics and Combustion} } 
\put(-0.75,2.97){\small \it \href{http://www.monash.edu.au/}{Monash University}, \href{http://www.melbourne.org/}{Melbourne}, \href{http://www.fed.gov.au/}{Australia}}
\put(-0.75,2.84){\small \it 9-10 December 1999 }
\end{picture}

\AFMCnormal
\noindent We have used Fourier transform techniques and an Abel deconvolution to analyse a finite-fringe interferogram produced by an axisymmetric shock wave flow, to produce a density map that can be used for the validation of a
numerical model. The Abel deconvolution method enables the
use of a basis that is particularly suitable for modeling
phase maps produced by shock wave flows. A steady flow
problem is studied, and compared with a numerical simulation. Good
agreement between theoretical and experimental results are obtained.

\noindent
{\\ \AFMChb INTRODUCTION}

\noindent Optical diagnostic techniques are very important for studying flows
produced in shock tubes and shock tunnels. The need to use
optically-based methods arises from the fact that the physical placement
of measurement probes within supersonic flows significantly perturbs the
flow and, in some cases, the variables one is trying to measure. Hence
the desire for non-intrusive techniques leads to the need to measure the
light that the flow radiates (emission, fluorescence etc), absorbs
(absorption), or deflects (schlieren, interferometry). In this paper, we
discuss a holographic interferometric method used to measure the density
distribution in an axially-symmetric flow in a shock tunnels for the purpose
of comparison with a numerical model.

\noindent Holography is a useful diagnostic for visualising and analysing a
variety of physical and engineering problems$^6$, while
holographic interferometry is particularly useful for density
measurements in flows produced by complex shock wave
configurations$^{13}$. In general, 
interferometry, using different techniques including Mach-Zehnder
interferometry, Michelson interferometry, and holographic
interferometry, has been very useful for providing data against which
computational fluid dynamic (CFD) simulation can be compared. In the
case of two-dimensional flows, comparisons between CFD and experiment is
reasonably straight-forward, since fringes in infinite-fringe
interferograms correspond directly to density contours$^8$.
In the case of axisymmetric or three-dimensional flows such fringes no
longer correspond to density contours and a direct comparison between
theoretical and experimental density is no longer possible. Under such
circumstances, post-processing of the CFD data is often performed to produce
theoretical interferograms against which the experimental ones can be
compared$^{4,9}$. This approach is also quite
successful. However, in all cases (two-dimensional, axisymmetric, and
three-dimensional), spurious contributions to the phase shift can
sometimes cause problems when attempting to interpret the data on
infinite-fringe interferograms. In addition, the spatial resolution is
limited by the fringe spacing,
which can be quite large when dealing with low density flows and/or
small optical path lengths. To overcome both of these difficulties,
finite-fringe interferometry, accompanied by a good interferogram
analysis technique, is arguably the best solution. 

\noindent In the case of three-dimensional flows, extracting density
information from the projected phase distributions is difficult because density
changes along the line-of-sight are averaged through the integration of
the refractive index along this direction. For complex three-dimensional
flows, tomographic reconstruction techniques are
required$^{12}$. In the case of axisymmetric flows, this
reconstruction reduces to an Abel deconvolution$^5$. In
this paper, we present the results from an experiment involving a flow that
possesses this symmetry property.

\noindent
{\\ \AFMChb FINITE FRINGE HOLOGRAPHIC INTERFEROMETRY}

\noindent The work discussed in this paper is based on the application of
finite-fringe double-exposure holographic
interferometry$^{13}$, a brief description of which is
presented in this section. A beam from a pulsed laser is split into two beams
by a beam-splitter to form the object and reference beams.
Infinite-fringe holograms are produced by keeping the optics for both
the reference and object beam fixed between exposures. If the reference
beam is tilted through a small angle between exposures, a finite-fringe
interferogram is produced. The first exposure is made before the test event, 
and the second exposure is made during the test event. For finite-fringe
interferometry the displacement of the reference beam
adds a linearly varying phase shift, that superimposes a
hetrodyning frequency over the whole image. 

\noindent
{\\ \AFMChb OVERVIEW OF INTERFEROGRAM ANALYSIS}


\noindent The interferogram analysis technique used
in the current work is a two-dimensional Fourier-based
method$^{2,3,1}$,
which consists of the following steps:
%
\begin{enumerate}
%
\item{apply the two-dimensional Fourier transform to the intensity
distribution in the interferogram
to produce both positive and negative frequency components};
%
\item{apply a filter operation in the Fourier transform plane
to remove noise and to select only the positive frequency
components};
%
\item{perform a frequency shift in the Fourier transform plane, so
that the data is located around the origin};
%
\item{apply the two-dimensional inverse Fourier transform to produce
real and imaginary parts of the frequency-shifted and filtered
intensity};
%
\item{determine the phase by evaluating the arctangent of the ratio of
the imaginary and real parts of the inverse transform};
%
\item{{\em `unwrap'} the phase by adding multiples of $2\pi$ where
required}; and
%
\item{remove any residual background phase}.
%
\end{enumerate}
%
The second step is required to remove both low and
high frequency noise from the data. By removing the negative frequency
components of the Fourier transform, this step also ensures that the
complex form of the intensity is produced through the application of the
inverse Fourier transform in step 4. The third step removes the carrier,
or hetrodyning, frequency. Of the above steps, the sixth is the most
difficult, and requires a sophisticated search algorithm$^3$. 

\noindent
{\\ \AFMChb DETERMINING THE DENSITY FROM THE UNWRAPPED PHASE}

\noindent In axisymmetric flows the projected phase $\phi(y,z)$ is the
summation of refractive index along the line of sight, which can be
represented by
%
\begin{equation}
\phi(y,z) - \phi_{r\! e\! f} =
\frac {2\pi}{\lambda}
\int [n_{r\! e\! f}-n_{f\! l\! o\! w}(\sqrt{x^2+y^2},z)]  \, dx \; , 
\label{eqn:phase_integral}
\end{equation}
%
where $dx$ is the incremental path length through the phase-shifting
medium. $\lambda$ is the wavelength of the light, and
$n$ is the refractive index. Here $y$ and $z$ are coordinates in an
$xyz$ Cartesian coordinate system, with $x$ in the direction of the
line-of-sight, and flow properties depend only on the values of $y$ and
$z$. The subscripts {\emph{ref}} and {\emph{flow}} refer to `reference'
and `flow' conditions, respectively. For a perfect gas of uniform
composition, the refractive index can be related directly to the
density, $\rho$, of the gas and its value at a standard
density$^{11}$, ${\rho_s}$,
%
\begin{equation}
n = 1 + \beta \frac{\rho}{\rho_s}
\label{eqn:beta}\; . %\eqno(8)
\end{equation}
%
Values for $\beta$ and $\rho_s$ for the gases used in the current work
are available in the literature$^{11}$. Using the above
equations, the density of the flow is determined to be given by
%
\begin{equation}
\rho _{f\! l\! o\! w}-\rho _{r\! e\! f}=
{{\lambda \rho _s} \over {2\pi W\beta }}
\left( {\phi _{r\! e\! f}-\phi _{f\! l\! o\! w}} \right) .
\label{eqn:density_2d}
\end{equation}
The integral in
Eq.~\ref{eqn:phase_integral} is the Abel transform$^5$ of
$\frac{\pi }{\lambda }\left[ {n_{r\! e\! f}  - n_{f\! l\! o\! w} \left( { \cdot ,z} \right)} \right]$.
Here, we will adopt a cylindrical polar coordinate
system, with the $z$-coordinate (axial coordinate) measured along the
axis of symmetry and $r$-coordinate (radial coordinate) measured along
radii centered on the axis of symmetry. The inverse Abel transform can be
used to write the radial
distribution of the refractive index in terms of the projected phase as
follows$^5$,
\begin{equation}
n_{r\! e\! f} - n_{f\! l\! o\! w}(r,z) =
\frac{-\lambda}{2\pi^2} \int_{r}^{\infty}[\frac{1}{\sqrt{y^2-r^2}}
\frac{d\phi(y,z)}{dy}] \, dy \; . 
\label{eqn:Abel1}
\end{equation}

\noindent Using Eqs.~\ref{eqn:beta}~and~\ref{eqn:Abel1}, it is possible to
determine the radial density distribution. In some cases, the phase data
can be fitted to a linear combination of basis functions for which the
analytic solutions to Eq.~\ref{eqn:Abel1} are known. The density is then
readily determined by the same linear combination of the deconvoluted
basis functions$^5$. This approach is a more efficient and,
often, a more accurate way of performing the deconvolution than through
direct numerical evaluation of the integral in Eq.~\ref{eqn:Abel1}. For
example, certain bases  are well-suited to the fitting of the
projected phase distributions resulting from the discontinuous jumps
across shock waves. This is the approach used in the current work.

\noindent We assume that the refractive index distribution, 
$n_{f\! l\! o\! w}(r,z) - n_{r\! e\! f}$ can be described
by a function $f\left( r, z \right)$, which, for a given value of $z$,
can be approximated by a linear combination of basis functions,
$f_i\left( r, z \right)$,
%
\begin{equation}
n_{f\! l\! o\! w}(r,z) - n_{r\! e\! f} =
f\left( r, z \right) \approx
\sum\limits_i {c_i\left(z \right) f_i\left( r, z \right)} \; ,
\label{eqn:f_rz_sum}
\end{equation}
%
where $c_i\left(z \right)$ are fitting coefficients for a particular
axial location $z$ and where the basis functions
$f_i\left( r, z \right)$ can depend also on the value of $z$.

\noindent The integral of the refractive index along the line of sight is then given
by$^5$
%
\begin{equation}
f_A\left( y, z \right)  \approx
\sum\limits_i {c_i\left(z \right) f_{A,i}\left( y, z \right)} \; ,
\label{eqn:f_A_rz_sum}
\end{equation}
%
where ${f_{A,i}\left( y, z \right)}$ are the Abel transforms of the basis
functions ${f_i\left( r, z \right)}$.
The type of density variations expected will influence the choice of basis,
while the number of functions depends on the required
accuracy, resolution of the original data and computational restrictions.
Here, we use five functions, which we will refer to as the
{\em ``well-suited''} basis,
$f_1\left( r, z \right), \dots, f_5\left( r, z \right)$, 
as defined below:
\begin{equation}
f_1\left( r, z \right)=\Pi \left( {r/ 2a(z)} \right) \; ;
\label{eqn:f1}
\end{equation}
\begin{equation}
f_2\left( r, z \right)=
\left( {a(z)^2-r^2} \right)^{-{1 \over 2}}\Pi \left( {r/ 2a(z)} \right) \; ;
\label{eqn:f2}
\end{equation}
\begin{equation}
f_3\left( r, z \right)=
\left( {a(z)^2-r^2} \right)^{{1 \over 2}}\Pi \left( {r/ 2a(z)} \right) \; ;
\label{eqn:f3}
\end{equation}
\begin{equation}
f_4\left( r, z \right)=
\left( {a(z)^2-r^2} \right)\Pi \left( {r/ 2a(z)} \right) \; ;
\label{eqn:f4}
\end{equation}
\begin{equation}
f_5\left( r, z \right)=
\left( {a(z)^2-r^2} \right)^{{3 \over 2}}\Pi \left( {r/ 2a(z)} \right) \; ;
\label{eqn:f5}
\end{equation}
where $a(z)$ is the radial position of the shock for a given value of $z$
and $\Pi \left( {\zeta} \right) = 1$ if $|\zeta| < \frac{1}{2}$
and
$\Pi \left( {\zeta} \right) = 0$ if $|\zeta| \ge \frac{1}{2}$.

\noindent Using Eq.~\ref{eqn:phase_integral}, the coefficients
$c_i\left(z\right)$ and the value of $a(z)$ at each value of $z$ are determined by
fitting the phase to the experimentally determined phase distribution,
$\phi \left( y, z \right) -\phi_{r\! e\! f}$,
according to the equation:
%
\begin{equation}
f_A\left( y, z \right) =
\frac {\lambda}{2\pi} 
\left[ \phi \left( y, z \right) -\phi_{r\! e\! f} \right]  \; .
\label{eqn:fitting}
\end{equation}
%
The values of $c_i\left(z\right)$ and $a(z)$ are determined by treating them as free
parameters in a least squares fitting routine that minimises the value of
$\sum\limits_j {
\left( f_A\left( y, z \right) - \frac {\lambda}{2\pi}
\left[ \phi \left( y, z \right) -\phi_{r\! e\! f} \right]
\right)^2_j}$,
where the summation is over the fitted data points for a given value of z.
Once these values are determined, $f\left( r, z \right)$ can be evaluated
from Eq.~\ref{eqn:f_rz_sum}, and the density is given by
%
$%\begin{equation}
\rho\left( r, z \right) = \frac{\rho_s}{\beta} f\left( r, z \right)
+ \rho_{r\! e\! f}  \; .
$%\end{equation}


\noindent
{\\ \AFMChb EXPERIMENT}

\noindent We studied a steady flow produced by a hypersonic
flow incident upon a stationary axisymmetric body. This experiment was
carried out in a free-piston shock
tunnel$^{10}$. The axisymmetric body used was a spherically-blunted
cylinder with a diameter of 50~mm, whose axis was aligned with the axis
of symmetry of the incident flow. The model was placed centrally within
the inviscid core flow downstream of the nozzle (130~mm exit diameter) of the shock tunnel. The inviscid core
was estimated from pitot pressure measurements to be approximately
100~mm in diameter.

\noindent The test gas was partially dissociated air,
the specific total enthalpy of the flow was 4.78~MJ/kg, and
the speed, density, static pressure, and static temperature in the
freestream were 2.75~km/s, 0.02~kg/m$^3$, 2.26~kPa, and 387~K,
respectively, with uncertainties of $\pm$~5\%.
The theoretical calculations used in the current work assume
that the flow remains in thermal equilibrium during the expansion
through the shock tunnel's nozzle. That is, the rotational and
vibrational temperatures are assumed equal. Based on this assumption, we
can approximate the supersonic flow at the exit of the nozzle as the
flow of an ideal gas with a ratio of specific heats of 1.33. The
assumption of thermal equilibrium is expected to result in systematic
errors in the calculated flow conditions as vibrational non-equilibrium
effects are expected to be important in the nozzle flow. 

\noindent
{\\ \AFMChb INTERFEROGRAM}

\noindent A finite fringe interferogram produced in the experiment is shown
in Fig.~\ref{fig:inter_hyper}.
A portion of an interferogram shows a continuous fringe shift across the
shock. This is a consequence of the axial symmetry of the flow. The
interferogram is the result of integration along the line of sight,
which produces a projection on the holograms
that effectively locates the shock position in a symmetry plane of the
flow, where the shock is at its most-upstream location. The optical path
length through the shock layer, directly behind the shock front in this
symmetry plane, is zero, and continuously increases with distance
downstream. Thus, even though there is a discontinuous density jump
across the shock, the phase shift changes continuously across the shock.
\begin{figure}[!h]
\begin{center}
{\sc
\resizebox{\picwidth}{!}{
\ifpdf
\includegraphics{sphere_inter_high_het.pdf}
\else
\includegraphics[158,258][452,533]{sphere_inter_high_het.eps}
\fi
}
}
\end{center}
\caption{A portion of an interferogram of hypersonic flow over a spherically-blunted cylinder.}
\label{fig:inter_hyper}
\end{figure}

\noindent
{\\ \AFMChb PHASE MAP}

\noindent The experimental phase
data for the flow is given in
Fig.~\ref{fig:spheretop_phasecor_image}.
This was generated using the finite-fringe interferogram analysis technique
described above.
This phase map is used to determine the flow density.
\begin{figure}[!h]
\begin{center}
{\sc
\resizebox{\mpicwidth}{!}{
\ifpdf
\includegraphics{spheretop_phasecor_image.pdf}
\else
\includegraphics[75,186][630,571]{spheretop_phasecor_image.eps}
\fi
}
}
\end{center}
\caption{Experimental phase map for flow over spherically-blunted cylinder.}
\label{fig:spheretop_phasecor_image}
\end{figure}

\noindent
{\\ \AFMChb FLOW DENSITY}

\noindent
{\AFMChb {Deconvolution with a well-suited basis}}

\noindent Here we analyse the data using the well-suited basis described
above. The first step in this
process involves fitting the experimental phase maps to the sum of basis
functions as given by Eq.~\ref{eqn:f_A_rz_sum}. 
To perform this we use a least-squares fitting algorithm. The
density maps determined from the deconvolution of the best fit
to the phase data is displayed in Fig.~\ref{fig:spheretop_density_image}.
Profiles along different cuts in this image is presented in
Fig.~\ref{fig:spheretop_density_cuts}.
The most notable feature of these results is that the sharp density jumps
across the shocks have been well resolved.
\begin{figure}[!h]
\begin{center}
{\sc
\resizebox{\spicwidth}{!}{
\ifpdf
\includegraphics{spheretop_density_image.pdf}
\else
\includegraphics[80,93][479,748]{spheretop_density_image.eps}
\fi
}
}
\end{center}
\caption{Density map for flow over spherically-blunted cylinder.}
\label{fig:spheretop_density_image}
\end{figure}
\begin{figure}[!h]
\begin{center}
{\sc
\resizebox{\picwidth}{!}{
\ifpdf
\includegraphics{spheretop_density_cuts.pdf}
\else
\includegraphics[26,74][572,701]{spheretop_density_cuts.eps}
\fi
}
}
\end{center}
\caption{Density profiles for flow over spherically-blunted cylinder.}
\label{fig:spheretop_density_cuts}
\end{figure}

\noindent Another observation is that, for some of the results, an overshoot
occurs near the shock front. This is related to the difficulty of
fitting the phase accurately close to the shock front, which is in turn
related to the uncertainty in locating the radial position $a(z)$ of the
shock front. This comes about as follows: the fitting algorithm used in
the current work experienced difficulties in iterating towards the best
value for the radial position $a(z)$ of the shock. It was found that
when $a(z)$ was used as a free parameter in the fitting algorithm, the
algorithm failed to converge to a solution for the fitting coefficients.
To overcome this problem, we modified the fitting technique so that the
shock position was found first. Because of the continuous change in the
projected phase across a shock in an axisymmetric flow, and because of
non-zero noise levels, there will be an error associated with finding
the value of $a(z) = r_{shock}(z)$. This error contributes to errors in
the fitted function, with the largest errors occurring within two or
three pixels of the shock front. This error will
sometimes cause an {\emph {overshoot}} in the value of the density at
the shock front, however, this overshoot rapidly decays back down to the
expected value of the post-shock density as one moves away from the
shock front.

\noindent
{\AFMChb {Comparison with deconvolution using a polynomial basis}}

\noindent Figure~\ref{fig:spheretop_density_cuts_compare}
compares density profiles obtained using fifth order polynomial basis
functions with those obtained using the well-suited basis.
From these comparisons, two distinct differences are immediately obvious.
Firstly, the deconvolution using the well-suited basis is more
successful at reproducing the sharp density rises across the shock than the
deconvolution using the polynomial basis. Secondly, the
deconvolution using the well-suited basis results in larger
values for the post-shock density than the deconvolution using the
polynomial basis. 
\begin{figure}[!h]
\begin{center}
{\sc
\resizebox{\picwidth}{!}{
\ifpdf
\includegraphics{spheretop_density_cuts_compare.pdf}
\else
\includegraphics[34,78][571,490]{spheretop_density_cuts_compare.eps}
\fi
}
}
\end{center}
\caption{Comparisons of profiles of density obtained using different bases for the hypersonic flow over spherically-blunted cylinder.}
\label{fig:spheretop_density_cuts_compare}
\end{figure}

\noindent These differences can be attributed to the spanning properties
of both basis sets, that is, the distinctive shapes of the functions involved.
The suitability of any particular basis can be indicated by firstly, the
overlap (inner product) of each basis function with the object in
question, and secondly, by the difference (orthogonality) of each basis
function with each other basis function. The first property ensures that
the basis has common features with the object, and the second property
ensures that each basis function is sufficiently different so to be able
to resolve as many of the object features as possible. In our flow, the
most notable feature is the shock wave, which can be broadly described
as a step function, hence the inclusion of the function in Eq.~\ref{eqn:f1} to the well-suited basis.
The polynomial basis does not include such a function, hence it is expected
that it will poorly resolve any step function, as indicated in
Fig.~\ref{fig:spheretop_density_cuts_compare}. With both bases the
functions involved are sufficiently different as to be able to resolve some
of the smaller features in the flow.

\noindent
{\\ \AFMChb NUMERICAL METHOD}

\noindent To solve the steady shock layer for the flow over
the spherically-blunted cylinder, we used the compressible flow solver,
CFD-FASTRAN$^{\textrm{TM}}$ $^7$. 
This solver is a finite volume, density-based
method, that can use either Roe, Van Leer, or flux splitting algorithms.
It is capable of implementing higher order differencing schemes
(up to third order), including Min-Mod, Osher-Chakravarthy, MUSCL, and Van Leer
limiters.  Different types of turbulence models
($k-\epsilon$, $k-\omega$, Baldwin-Lomax) can be used, and a number
of time integration schemes can be adopted.  However, for
the case described here, only steady inviscid calculations were made
for the flow. The result of the calculation is shown in
Fig.~\ref{fig:hemi_theory}.
\begin{figure}[!h]
\begin{center}
{\sc
\resizebox{\spicwidth}{!}{
\ifpdf
\includegraphics{hemi.pdf}
\else
\includegraphics[2,110][617,731]{hemi.eps}
\fi
}
}
\end{center}
\caption{Theoretical density results for steady flow over spherically-blunted cylinder. Calculation performed using the code CFD-FASTRAN$^{\textrm{TM}}$ $^7$.}
\label{fig:hemi_theory}
\end{figure}

\noindent
{\\ \AFMChb COMPARISON OF CFD AND EXPERIMENTAL RESULTS}

\noindent Figure~\ref{fig:spheretop_density_cuts_new}
shows theoretical and experimental density profiles for the flow.
Profiles are presented for axial cuts at different radial positions.
Apart from the very large overshoot close to the shock vertex, agreement
between theory and experiment is very good. Cuts at $r$ = 5 and 10~mm are
close to the stagnation region, where the density is essentially constant.
Cuts at $r$ = 20, 25 and 30~mm are through both the shock wave and the
expansion wave near the shoulder of the model, for which the density
rises sharply across the shock and decreases continuously through the
expansion wave. These trends are observed in both the CFD and
experimental results and the CFD code correctly predicts the shock
standoff distance from the model surface, providing confidence in the
accuracy of the CFD modeling for these conditions. Quantitative
agreement between the measured and calculated density is also very good.
\begin{figure}[!h]
\begin{center}
{\sc
\resizebox{\picwidth}{!}{
\ifpdf
\includegraphics{spheretop_density_cuts_new.pdf}
\else
\includegraphics[-61,34][562,745]{spheretop_density_cuts_new.eps}
\fi
}
}
\end{center}
\caption{Density profiles for flow over spherically-blunted cylinder: theoretical and experimental results.}
\label{fig:spheretop_density_cuts_new}
\end{figure}

\noindent Based on the uncertainties in the measured primary shock speed
and the nozzle reservoir pressure, we expect a systematic uncertainty of
approximately $\pm$~5\% in the CFD-calculated density distribution.
Provided sufficient data points are available for accurate fitting to
the phase distribution, and neglecting the overshoot problem close to
the shock front, random and systematic errors from the interferometric
measurements of density are estimated to be less than 1\%. However,
within 2 or 3 pixels (less than 1~mm) of the shock front, errors of the
order of 100\% can sometimes be encountered due to the overshoot problem,
as discussed above.

\noindent
{\\ \AFMChb CONCLUSIONS}

\noindent We have demonstrated the use of a well-suited basis, to
deconvolute shock wave phase data produced in an axisymmetric flow, to
determine the density. The method has
been successful in resolving the discontinuous density changes across
the shock wave in the flow studied. A CFD simulation has
been discussed and the result from this simulation has been successfully
compared with the experimental result. The good quantitative agreement
between the CFD and experimental result gives confidence in the
validity of the CFD model investigated.

\noindent
{\\ \AFMChb ACKNOWLEDGMENTS}

\noindent We thank O.~Onodera, T.~Ogawa and H.~Ojima for
their technical and professional expertise in assisting to conduct the
experiment described in this paper. We also express our appreciation to
H.~Babinsky for making available the source listings of his
algorithms on which part of the work reported here is based. 
\href{mailto:Frank.Houwing@anu.edu.au}{A.~F.~P. Houwing} gratefully
acknowledges the support given to
him as Visiting Professor at the
\href{http://ceres.ifs.tohoku.ac.jp/~swrc/}{Shock Wave Research Center} at the
\href{http://ifs.tohoku.ac.jp/}{Institute of Fluid Science} at
\href{http://www.tohoku.ac.jp/}{Tohoku University} in
\href{http://www.city.sendai.jp/index-e.html}{Sendai} to participate
in the collaborative research project described in this paper. 

\ 

\noindent
{\\ \AFMChb  REFERENCES}

\noindent
$^1$\hspace{1pt} BABINSKY, H. and TAKAYAMA, K., ``Quantitative holographic interferometry of shock-wave flows using Fourier transform fringe analysis'',
{\it Proc. 20th Intn. Sym. Shock Waves}, {\it Pasadena July 1995}, {\it World Scientific Press}, 1599-1604, 1995

\noindent
$^2$\hspace{1pt} BONE, D.~J., BACHOR, H.~A., and SANDEMAN, R.~J., ``Fringe-pattern analysis using a 2-D Fourier transform'',
{\it Applied Optics}, {\bf 25}(10), 1653-1660, 1986.

\noindent
$^3$\hspace{1pt} BONE, D.~J., ``Fourier fringe analysis -- the two-dimensional phase unwrapping problem'',
{\it Applied Optics}, {\bf 30}(25), 3627-3632, 1991.

\noindent
$^4$\hspace{1pt} BOYCE, R.~R.,MORTON, J.~W., HOUWING, A.~F.~P., MUNDT, Ch., and BONE, D.~J., ``CFD validation using multiple interferometric views of three-dimensional shock layer flows over a blunt body'', 
{\it Journal of Spacecraft and Rockets}, {\bf 33}(3), 319--325, 1994

\noindent
$^5$\hspace{1pt} BRACEWELL, R.~N., ``The Fourier transform and its applications'', 
{\it New York: McGraw-Hill Book Company}, 264, 1965.

\noindent
$^6$\hspace{1pt} CAUFIELD, H.~J., ``Handbook of optical holography'',
{\it Academic Press}, 475, 1979.

\noindent
$^7$\hspace{1pt} CFD-FASTRAN$^{\textrm{TM}}$,
{\it CFD Research Corporation}, \href{http://www.cfdrc.com/}{http://www.cfdrc.com/}.

\noindent
$^8$\hspace{1pt} HORNUNG, H.~G., ``Non-equilibrium dissociating nitrogen flow over spheres and circular cylinders'',
{\it J. Fluid Mech.}, {\bf 53}, 149--176, 1972.

\noindent
$^9$\hspace{1pt} JIANG, Z., ONODERA, O., and TAKAYAMA, K., ``Evolution of shock waves and the primary vortex loop discharged from a square cross-sectional tube'',
{\it Shock Wave Journal}, {\bf 9}, 1--10, 1999.

\noindent
$^{10}$\hspace{1pt} KOREMOTO, K., ``Optimization of the characteristics of a free-piston shock tunnel'',
{\it PhD Thesis}, {\it Tohoku University}, 1999.

\noindent
$^{11}$\hspace{1pt} LIEPMANN, H.~W. and ROSHKO, A., ``Elements of gasdynamics'',
{\it John Wiley and Sons Inc}, 154, 1957.

\noindent
$^{12}$\hspace{1pt} MORTON, J.~W., HOUWING, A.~F.~P., BOYCE, R.~R., and BONE, D.~J., ``Tomographic reconstruction of jet and shock layer flows'',
{\it Proc. 21st International Symposium on Shock waves}, {\it Australia}, {\it Panther Printing and Publishing Canberra}, 435--440, 1998.

\noindent
$^{13}$\hspace{1pt} TAKAYAMA, K., ``Application of holographic interferometry to shock wave research'',
{\it Proc. of the Society of Photo-Optical Instrumentation Engineers}, {
\bf 398}, 174--180, 1983.

\noindent
$^{14}$\hspace{1pt} VARDAVAS, I.~M., ``Modeling reactive gas flows within shock tunnels'',
{\it Aust. J. Phys.}, {\bf 37}, 157--177, 1984.
%%%%%%%%%%%%%%%%%%%%%%%%%%%%%%%%%%%%%%%%%%%%%%%%%%%%%%%%%%%%%%%%
\end{document}
