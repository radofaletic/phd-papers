\documentclass[a4paper]{article}
\usepackage{graphics}
\pagestyle{plain}
\font\AFMCcl=cmssbx10 
\font\AFMCab=cmssbx10
\font\AFMCan=cmss10
\font\AFMChb=cmssbx10 scaled 900
\font\AFMCfg=cmss10 scaled 900
\font\AFMCnormal=cmr9
\font\AFMCfoot=cmr8
\setlength{\unitlength}{1in}

\begin{document}

\noindent
{\AFMChb {Deconvolution with ill-suited basis functions}}

\noindent To demonstrate the importance of choosing the appropriate basis
functions for the deconvolution process, we first analyse the data
without using the basis functions defined in
Eqs.~\ref{eqn:fA1}~to~\ref{eqn:fA5}. This can be achieved by directly
applying Eqs.~\ref{eqn:beta}~and~\ref{eqn:Abel1} to determine the radial
density distribution. The integrand in Eq.~\ref{eqn:Abel1} contains a
derivative, $\frac{d\phi}{dy}$, which results in large errors when
evaluated numerically. To overcome this problem, this derivative is
determined from a fifth-order polynomial fit to the data. The polynomial
smoothing function is equivalent to using power-law basis functions in
Eq.~\ref{eqn:f_A_rz_sum}, while the application of Eq.~\ref{eqn:Abel1}
is essentially determining the inverse Abel transforms of these
power-law basis functions (or the basis functions of the inverse Abel
transform) numerically.

\noindent Power-law basis functions are not particularly well-suited for
describing phase distributions caused by shock waves, because such
distributions contain discontinuities in their first derivatives, while
the power-law basis functions are infinitely differentiable. As a
result, it is expected that such an approach will result in a failure to
accurately represent the discontinuities in density, leading to
`{\em smeared}' shock fronts.

\noindent The result from analysing the flow using power-law basis
functions is shown in
Figure~\ref{fig:density_map_no_basis}.
\begin{figure}[!h]
\vspace{15mm}
\begin{center}
{\sc %\epsfxsize=0.7\columnwidth 
     %\epsfbox{spheretop_density_image.eps}
\resizebox{.5\textwidth}{!}{\includegraphics{spheretop_density_image.eps}}
}
\end{center}
\vspace{15mm}
\caption{Density map obtained using power-law basis functions for the hypersonic flow over spherically-blunted cylinder.}
\label{fig:density_map_no_basis}
\end{figure}
%\noindent {\bf Figure 5:} Density map obtained using power-law basis functions for the hypersonic flow over spherically-blunted cylinder.
Density profiles along different cuts through the density image are shown in
Figure~\ref{fig:spheretop_dens_p2}.
\begin{figure}[!h]
\vspace{15mm}
\begin{center}
{\sc %\epsfxsize=\columnwidth 
     %\epsfbox{spheretop_density_p2.eps}
\resizebox{.5\textwidth}{!}{\includegraphics{spheretop_density_p2.eps}}
}
\end{center}
\vspace{15mm}
\caption{Profiles of density obtained using power-law basis functions for the steady flow over the spherically-blunted cylinder.}
\label{fig:spheretop_dens_p2}
\end{figure}
%\noindent {\bf Figure 6:} Profiles of density obtained using power-law basis functions for the steady flow over the spherically-blunted cylinder.
In these profiles, one can observe the
density jumps across the shock waves as well as the density changes
associated with the expansion waves. However, as expected, the Abel
deconvolution has not been entirely successful in producing the steep
density jump across the shock wave. The density increase should be a
step function across the shock, however, in the results shown in
Figure~\ref{fig:spheretop_dens_p2}, the deconvolution has `smeared' the
shock wave. Nonetheless, the deconvoluted results show the qualitative
features of the flow quite well: the density rises across the shock and
the density decreases continuously in regions where the flow is expected
to expand. 

\end{document}
